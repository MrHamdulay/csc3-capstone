%%%%%%%%%%%%%%%%%%%%%%%%%%%%%%%%%%%%%%%%%%%%%%%%%%%%%%%%%%%%%%%%%
%%%  Capstone Project Template that tries to save a few trees %%%
%%%  Edwin Blake 22 Aug 2013                                                   %%%
%%%		1 Aug 2014 (revised)                                      %%% 
%%%  see also                                                                             %%%
%%% http://ravirao.wordpress.com/2005/11/19/latex-tips-to-meet-publication-page-limits/  
%%%%%%%%%%%%%%%%%%%%%%%%%%%%%%%%%%%%%%%%%%%%%%%%%%%%%%%%%%%%%%%%%

\documentclass[11pt,a4paper]{article}
\usepackage{times}
\usepackage{fancyhdr}           % Allows better control over headers and footers
%\usepackage{layout}            % use with \layout to see the page layout for
%debugging purposes.
\usepackage[margin=2.5cm]{geometry}  %   set the margins using the
                                %   geometry package (which is much
                                %   the easiest way of doing this).
\usepackage[pdftex]{graphicx}   %   Pictures (means you have to
                                %   produce pdf output via pdflatex)
\usepackage[small,compact]{titlesec}   % Try to reduce the white space
                                % latex loves so much
\titlelabel{\thetitle. \quad}   % Reduce space around section heads
                                % and add a full stop after the number
\pagestyle{fancy}               % Invoke fancy headers

\renewcommand{\abstractname}{\vskip -5mm}  %  Change name of Abstract
                                %  to nothing and loose some of the
                                %  excessive white space


\begin{document}

\title{Cheaters Plagiarism Detection Report} \date{}
\author{Merishka Lalla\\joe@bloggs.com
\and Yaseen Hamdulay\\jane.doe@uct.ac.za
\and Jarred de Beer\\john.doe@uct.ac.za}

%%%  Set the headers via fancyhdr package
\lhead{Capstone Project Report}  % Short title for running head
\chead{}
\rhead{1st August, 2014}   %  Fixed running head of the date
\lfoot{}
\cfoot{\thepage}    %  add page number as centre footer.
\rfoot{}
\renewcommand{\headrulewidth}{0.0pt}   % Don't want horizontal line
                                % under header.

\maketitle
\thispagestyle{plain}  % First page is plain style headings and
                       % footers (ie just the page number as footer).

\section*{Abstract}
First you should have an executive summary (or abstract) just a single
paragraph saying what the results of the project are (at most 200
words).

\section{Introduction}
\label{s:introduction}

Your introduction provides the context for the project and should
contain the statement of the scope of the project (which may have
changed since you first wrote it). Someone reading your introduction
must have clear idea of what the system is intended for. If you think
there is something special about the kind of problem you tackled that
your reader needs to know up front then this is where you say it.

If you need any survey of other work (you probably don't) then put it
towards the end of the introduction and give suitable references. A
case where this is needed is if your project builds on someone else's
project or some published algorithm.

Discuss your approach to solving the problem. Please give a short
overview of the software engineering methods you used (e.g.,
traditional analysis followed by design and implementation -- typically
the case if you did an evolutionary prototype, or a more agile
approach where you had a cyclical development process). 

\section{Requirements Captured}

The next section deals with the analysis of your system. Cover the
functional, non-functional and usability requirements. This is where
you present your use case narratives and diagrams. 

Discuss the major analysis artefacts that you produced. We will expect
you to produce at least one overall description of the architecture
used in your system as a diagram, either here or below (see Section
\ref{s:design-overview}). You may also want to include an analysis
class hierarchy diagram.

\section{Design Overview}
\label{s:design-overview}

The next section is an overview of your design. The system design has
to be justified in terms of the expected behaviour of the final
product. 

If you produced a design class diagram put it here.

\begin{figure}[h!]
  \center{\includegraphics[scale=0.8]{architecture.png}}
  \caption{An architecture diagram. Caption to go below figure}
  \label{fig:architecture}
\end{figure}

You must present the overall architecture of the system together with
an architecture diagram. You may choose what kind of diagram best
suits your project but we would expect a layered architecture diagram
(see Figure \ref{fig:architecture}) unless there is a good reason for
some other kind of diagram. It need not be a formal UML diagram as
long as it conveys all the necessary information clearly.

You should then (in subsections) cover the algorithms and the data
organisation used and why they were considered the best. 

\section{Implementation}

Now we get to the details. 

\begin{itemize}
\item Describe your data structures and be sure to illustrate them
  with a diagram.

\item If your user interface was a key feature describe how that was
  implemented.

\item Discuss the function of the most significant methods in each
  class. This may well require flowcharts, or sequence diagrams, in
  some cases.

\item Any special relationship between the classes (e.g. friends) and
  why they exist.

\item A description of any special programming techniques or libraries
  used.
\end{itemize}

\section{Program Validation and Verification}
\label{ss:progr-valid-verif}

Tell us how you tested the system and why you believe it works.
Describe all the steps taken to validate the correctness of the
program.

If you had user tests then say what you did and what the results
were. Describe why these test data were chosen (what test conditions
the data was testing).  Table \ref{tab:tests} provides an example of
the sorts of results we are looking for. The full detail of the test
runs should be appended to the report.

\begin{table}[h!]
  \centering
\caption{A table of tests. A table caption goes above the table.}

  \begin{tabular}[t]{|p{5cm}|p{3cm}|p{3cm}|p{3cm}|} \hline \textbf{Data Set
    and reason for its choice} & \multicolumn{3}{c|}{\textbf{Test Cases}}\\
    \cline{2-4} & \emph{Normal Functioning} & \emph{Extreme boundary cases} &
    \emph{Invalid Data (program should not crash)} \\ \hline Preliminary test
    (see Appendix 3) & Passed & n/a & Fell over \\\hline &&&\\ \hline
    &&&\\ \hline
  \end{tabular}

\label{tab:tests}
\end{table}

Follow your table of results with a discussions of them highlighting
how useful and usable your system is for its intended purpose.

\section{Conclusion}
\label{ss:conclusion}

Your report must have a clear conclusion where you revisit the aims
set out in the beginning and discuss how well you met them. Did you
achieve the objective of creating a well-structured, modular, and
robust system?  Please summarize the design features and test results
that show this.

\begin{thebibliography}{9}

\bibitem[Kopka and Daly(2004)]{KopkaDaly}
Kopka, H. and Daly, P.W.  (2004) \textit{A Guide to \LaTeXe:
Document Preparation for Beginners and Advanced Users} (4th~edn).
Addison-Wesley.

\bibitem[Lamport(1994)]{Lamport}
Lamport L. (1994) \textit{\LaTeX: A Document Preparation System}
(2nd~edn). Addison-Wesley.

\bibitem[Mittelbach and Goossens(2004)]{Companion}
Mittelbach, F. and Goossens, M., (2004) \textit{The \LaTeX\
Companion} (2nd~edn). Addison-Wesley.

\end{thebibliography}
\end{document}
